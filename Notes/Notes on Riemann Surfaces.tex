\documentclass[12pt]{article}
\usepackage{amsmath}
\usepackage{enumerate}
\usepackage{amsthm}
\usepackage{amsfonts}
\usepackage{amssymb}
\usepackage{latexsym} 
%\usepackage{epsfig}
%\usepackage{graphicx}
%\usepackage[dvips]{graphicx}
\usepackage{tikz}
\usepackage{tikz-cd}



\usepackage[matrix,tips,graph,curve]{xy}

\newcommand{\mnote}[1]{${}^*$\marginpar{\footnotesize ${}^*$#1}}
\linespread{1.065}

\makeatletter

\setlength\@tempdima  {5.5in}
\addtolength\@tempdima {-\textwidth}
\addtolength\hoffset{-0.5\@tempdima}
\setlength{\textwidth}{5.5in}
\setlength{\textheight}{8.75in}
\addtolength\voffset{-0.625in}

\makeatother

\makeatletter 
\@addtoreset{equation}{section}
\makeatother


\renewcommand{\theequation}{\thesection.\arabic{equation}}

\theoremstyle{plain}
\newtheorem{theorem}[equation]{Theorem}
\newtheorem{corollary}[equation]{Corollary}
\newtheorem{lemma}[equation]{Lemma}
\newtheorem{proposition}[equation]{Proposition}
\newtheorem{conjecture}[equation]{Conjecture}
\newtheorem{fact}[equation]{Fact}
\newtheorem{facts}[equation]{Facts}
\newtheorem*{theoremA}{Theorem A}
\newtheorem*{theoremB}{Theorem B}
\newtheorem*{theoremC}{Theorem C}
\newtheorem*{theoremD}{Theorem D}
\newtheorem*{theoremE}{Theorem E}
\newtheorem*{theoremF}{Theorem F}
\newtheorem*{theoremG}{Theorem G}
\newtheorem*{theoremH}{Theorem H}

\theoremstyle{definition}
\newtheorem{definition}[equation]{Definition}
\newtheorem{definitions}[equation]{Definitions}
%\theoremstyle{remark}

\newtheorem{remark}[equation]{Remark}
\newtheorem{remarks}[equation]{Remarks}
\newtheorem{exercise}[equation]{Exercise}
\newtheorem{example}[equation]{Example}
\newtheorem{examples}[equation]{Examples}
\newtheorem{notation}[equation]{Notation}
\newtheorem{question}[equation]{Question}
\newtheorem{assumption}[equation]{Assumption}
\newtheorem*{claim}{Claim}
\newtheorem{answer}[equation]{Answer}
%%%%%% letters %%%%

\newcommand{\fA}{\mathfrak{A}}
\newcommand{\fB}{\mathfrak{B}}
\newcommand{\fC}{\mathfrak{C}}
\newcommand{\fD}{\mathfrak{D}}
\newcommand{\fE}{\mathfrak{E}}
\newcommand{\fF}{\mathfrak{F}}
\newcommand{\fG}{\mathfrak{G}}
\newcommand{\fH}{\mathfrak{H}}
\newcommand{\fI}{\mathfrak{I}}
\newcommand{\fJ}{\mathfrak{J}}
\newcommand{\fK}{\mathfrak{K}}
\newcommand{\fL}{\mathfrak{L}}
\newcommand{\fM}{\mathfrak{M}}
\newcommand{\fN}{\mathfrak{N}}
\newcommand{\fO}{\mathfrak{O}}
\newcommand{\fP}{\mathfrak{P}}
\newcommand{\fQ}{\mathfrak{Q}}
\newcommand{\fR}{\mathfrak{R}}
\newcommand{\fS}{\mathfrak{S}}
\newcommand{\fT}{\mathfrak{T}}
\newcommand{\fU}{\mathfrak{U}}
\newcommand{\fV}{\mathfrak{V}}
\newcommand{\fW}{\mathfrak{W}}
\newcommand{\fX}{\mathfrak{X}}
\newcommand{\fY}{\mathfrak{Y}}
\newcommand{\fZ}{\mathfrak{Z}}

\newcommand{\fa}{\mathfrak{a}}
\newcommand{\fb}{\mathfrak{b}}
\newcommand{\fc}{\mathfrak{c}}
\newcommand{\fd}{\mathfrak{d}}
\newcommand{\fe}{\mathfrak{e}}
\newcommand{\ff}{\mathfrak{f}}
\newcommand{\fg}{\mathfrak{g}}
\newcommand{\fh}{\mathfrak{h}}
%\newcommand{\fi}{\mathfrak{i}}

\newcommand{\fj}{\mathfrak{j}}
\newcommand{\fk}{\mathfrak{k}}
\newcommand{\fl}{\mathfrak{l}}
\newcommand{\fm}{\mathfrak{m}}
\newcommand{\fn}{\mathfrak{n}}
\newcommand{\fo}{\mathfrak{o}}
\newcommand{\fp}{\mathfrak{p}}
\newcommand{\fq}{\mathfrak{q}}
\newcommand{\fr}{\mathfrak{r}}
\newcommand{\fs}{\mathfrak{s}}
\newcommand{\ft}{\mathfrak{t}}
\newcommand{\fu}{\mathfrak{u}}
\newcommand{\fv}{\mathfrak{v}}
\newcommand{\fw}{\mathfrak{w}}
\newcommand{\fx}{\mathfrak{x}}
\newcommand{\fy}{\mathfrak{y}}
\newcommand{\fz}{\mathfrak{z}}


\newcommand{\sA}{\mathcal{A}\,}
\newcommand{\sB}{\mathcal{B}\,}
\newcommand{\sC}{\mathcal{C}}
\newcommand{\sD}{\mathcal{D}\,}
\newcommand{\sE}{\mathcal{E}\,}
\newcommand{\sF}{\mathcal{F}\,}
\newcommand{\sG}{\mathcal{G}\,}
\newcommand{\sH}{\mathcal{H}}
\newcommand{\sI}{\mathcal{I}\,}
\newcommand{\sJ}{\mathcal{J}\,}
\newcommand{\sK}{\mathcal{K}\,}
\newcommand{\sL}{\mathcal{L}\,}
\newcommand{\sM}{\mathcal{M}\,}
\newcommand{\sN}{\mathcal{N}}
\newcommand{\sO}{\mathcal{O}}
\newcommand{\sP}{\mathcal{P}\,}
\newcommand{\sQ}{\mathcal{Q}\,}
\newcommand{\sR}{\mathcal{R}}
\newcommand{\sS}{\mathcal{S}}
\newcommand{\sT}{\mathcal{T}\,}
\newcommand{\sU}{\mathcal{U}\,}
\newcommand{\sV}{\mathcal{V}\,}
\newcommand{\sW}{\mathcal{W}\,}
\newcommand{\sX}{\mathcal{X}\,}
\newcommand{\sY}{\mathcal{Y}\,}
\newcommand{\sZ}{\mathcal{Z}\,}

\newcommand{\IA}{\mathbb{A}}
\newcommand{\IB}{\mathbb{B}}
\newcommand{\IC}{\mathbb{C}}
\newcommand{\ID}{\mathbb{D}}
\newcommand{\IE}{\mathbb{E}}
\newcommand{\IF}{\mathbb{F}}
\newcommand{\IG}{\mathbb{G}}
\newcommand{\IH}{\mathbb{H}}
\newcommand{\II}{\mathbb{I}}
\newcommand{\IK}{\mathbb{K}}
\newcommand{\IL}{\mathbb{L}}
\newcommand{\IM}{\mathbb{M}}
\newcommand{\IN}{\mathbb{N}}
\newcommand{\IO}{\mathbb{O}}
\newcommand{\IP}{\mathbb{P}}
\newcommand{\IQ}{\mathbb{Q}}
\newcommand{\IR}{\mathbb{R}}
\newcommand{\IS}{\mathbb{S}}
\newcommand{\IT}{\mathbb{T}}
\newcommand{\IU}{\mathbb{U}}
\newcommand{\IV}{\mathbb{V}}
\newcommand{\IW}{\mathbb{W}}
\newcommand{\IX}{\mathbb{X}}
\newcommand{\IY}{\mathbb{Y}}
\newcommand{\IZ}{\mathbb{Z}}


 \newcommand{\tA}{\mathrm {A}}
 \newcommand{\tB}{\mathrm {B}}
 \newcommand{\tC}{\mathrm {C}}
 \newcommand{\tD}{\mathrm {D}}
 \newcommand{\tE}{\mathrm {E}}
 \newcommand{\tF}{\mathrm {F}}
 \newcommand{\tG}{\mathrm {G}}
 \newcommand{\tH}{\mathrm {H}}
 \newcommand{\tI}{\mathrm {I}}
 \newcommand{\tJ}{\mathrm {J}}
 \newcommand{\tK}{\mathrm {K}}
 \newcommand{\tL}{\mathrm {L}}
 \newcommand{\tM}{\mathrm {M}}
 \newcommand{\tN}{\mathrm {N}}
 \newcommand{\tO}{\mathrm {O}}
 \newcommand{\tP}{\mathrm {P}}
 \newcommand{\tQ}{\mathrm {Q}}
 \newcommand{\tR}{\mathrm {R}}
 \newcommand{\tS}{\mathrm {S}}
 \newcommand{\tT}{\mathrm {T}}
 \newcommand{\tU}{\mathrm {U}}
 \newcommand{\tV}{\mathrm {V}}
 \newcommand{\tW}{\mathrm {W}}
 \newcommand{\tX}{\mathrm {X}}
 \newcommand{\tY}{\mathrm {Y}}
 \newcommand{\tZ}{\mathrm {Z}}
%%%%%%% macros %%%%%

%% my definitions %%%

\newcommand{\End}{\mathrm{End}}
\newcommand{\tr}{\mathrm{tr}}
%\newcommand{\ind}{\mathrm{ind}}

\renewcommand{\index}{\mathrm{index \,}}
\newcommand{\Hom}{\mathrm{Hom}}
\newcommand{\Aut}{\mathrm{Aut}}
\newcommand{\Trace}{\mathrm{Trace}\,}
\newcommand{\Res}{\mathrm{Res}\,}
\newcommand{\rank}{\mathrm{rank}}
%\renewcommand{\dim}{\mathrm{dim}}

\renewcommand{\deg}{\mathrm{deg}}
\newcommand{\spin}{\rm Spin}
\newcommand{\Spin}{\rm Spin}
\newcommand{\erfc}{\rm erfc\,}
\newcommand{\sgn}{\rm sgn\,}
\newcommand{\Spec}{\rm Spec\,}
\newcommand{\diag}{\rm diag\,}
\newcommand{\Fix}{\mathrm{Fix}}
\newcommand{\Ker}{\mathrm{Ker \,}}
\newcommand{\Coker}{\mathrm{Coker \,}}
\newcommand{\Sym}{\mathrm{Sym \,}}
\newcommand{\Span}{\mathrm{Span \,}}
\newcommand{\Hess}{\mathrm{Hess \,}}
\newcommand{\grad}{\mathrm{grad \,}}
\newcommand{\Center}{\mathrm{Center}}
\newcommand{\Lie}{\mathrm{Lie}}


\newcommand{\ch}{\rm ch} % Chern Character

\newcommand{\rk}{\rm rk} 
%\renewcommand{\c}{\rm c}  % Chern class

\newcommand{\sign}{\rm sign}
\renewcommand\dim{{\rm dim\,}}
\renewcommand\det{{\rm det\,}}
\newcommand{\dimKrull}{{\rm Krulldim\,}}
\newcommand\Rep{\mathrm{Rep}}
\newcommand\Hilb{\mathrm{Hilb}}
\newcommand\vol{\mathrm{vol}}

\newcommand\QED{\hfill $\Box$} %{\bf QED}} 

\newcommand\Pf{\nonintend{\em Proof. }}


\newcommand\reals{{\mathbb R}} 
\newcommand\complexes{{\mathbb C}}
\renewcommand\i{\sqrt{-1}}
\renewcommand\Re{\mathrm Re}
\renewcommand\Im{\mathrm{Im} \,}
\newcommand\integers{{\mathbb Z}}
\newcommand\quaternions{{\mathbb H}}


\newcommand\iso{{\, \cong \,}} 
\newcommand\tensor{{\otimes}}
\newcommand\Tensor{{\bigotimes}} 
\newcommand\union{\bigcup} 
\newcommand\onehalf{\frac{1}{2}}
%\newcommand\Sym[1]{{Sym^{#1}(\complexes^2)}}

\newcommand\lie[1]{{\mathfrak #1}} 
\renewcommand\fk{\mathfrak{K}}
\newcommand\smooth{\mathcal{C}^{\infty}}
\newcommand\trivial{{\mathbb I}}
\newcommand\widebar{\overline}

%%%%%Delimiters%%%%

\newcommand{\<}{\langle}
\renewcommand{\>}{\rangle}

%\renewcommand{\(}{\left(}
%\renewcommand{\)}{\right)}


%%%% Different kind of derivatives %%%%%

\newcommand{\delbar}{\bar{\partial}}
\newcommand{\pdu}{\frac{\partial}{\partial u}}
%\newcommand{\pd}[1][2]{\frac{\partial #1}{\partial #2}}

%%%%% Arrows %%%%%
%\renewcommand{\ra}{\rightarrow}                   % right arrow
%\newcommand{\lra}{\longrightarrow}              % long right arrow
%\renewcommand{\la}{\leftarrow}                    % left arrow
%\newcommand{\lla}{\longleftarrow}               % long left arrow
%\newcommand{\ua}{\uparrow}                     % long up arrow
%\newcommand{\na}{\nearrow}                      %  NE arrow
%\newcommand{\llra}[1]{\stackrel{#1}{\lra}}      % labeled long right arrow
%\newcommand{\llla}[1]{\stackrel{#1}{\lla}}      % labeled long left arrow
%\newcommand{\lua}[1]{\stackrel{#1}{\ua}}      % labeled  up arrow
%\newcommand{\lna}[1]{\stackrel{#1}{\na}}      % labeled long NE arrow

\newcommand{\into}{\hookrightarrow}
\newcommand{\tto}{\longmapsto}
\def\llra{\longleftrightarrow}

\def\d/{/\mspace{-6.0mu}/}
\newcommand{\git}[3]{#1\d/_{\mspace{-4.0mu}#2}#3}
\newcommand\zetahilb{\zeta_{{\text{Hilb}}}}
\def\Fy{\sF \mspace{-3.0mu} \cdot \mspace{-3.0mu} y}
\def\tv{\tilde{v}}
\def\tw{\tilde{w}}
\def\wt{\widetilde}
\def\wtilde{\widetilde}
\def\what{\widehat}
\def\cl{\widebar}
\def\bf{\textbf}
\def\it{\textit}
%%%%%%%%%%%%%%%%%%% Mark's definitions %%%%%%%%%%%%%%%%%%%%

\newcommand{\frakg}{\mbox{\frakturfont g}}
\newcommand{\frakk}{\mbox{\frakturfont k}}
\newcommand{\frakp}{\mbox{\frakturfont p}}
\newcommand{\q}{\mbox{\frakturfont q}}
\newcommand{\frakn}{\mbox{\frakturfont n}}
\newcommand{\frakv}{\mbox{\frakturfont v}}
\newcommand{\fraku}{\mbox{\frakturfont u}}
\newcommand{\frakh}{\mbox{\frakturfont h}}
\newcommand{\frakm}{\mbox{\frakturfont m}}
\newcommand{\frakt}{\mbox{\frakturfont t}}
\newcommand{\G}{\Gamma}
\newcommand{\g}{\gamma}
\newcommand{\w}{\omega}
\newcommand{\mult}{\mathrm{mult}}
\newcommand{\ord}{\mathrm{ord}}
\newcommand{\fraka}{\mbox{\frakturfont a}}
\newcommand{\bp}{\overline{\partial}}
\newcommand{\dbs}{\bar{\partial}^*}
\newcommand{\p}{\partial}
\newcommand{\id}{\mathrm{id}}
\newcommand{\bi}{\mathbf{1}}
\newcommand{\Ann}{\mathrm{Ann}}
\newcommand{\height}{\mathrm{height \,}}

\newcommand{\half}{\frac{1}{2}}
\newcommand{\bz}{\overline{z}}
\newcommand{\bw}{\overline{w}}
\newcommand{\Div}{\mathrm{Div}}
\newcommand{\PDiv}{\mathrm{PDiv}}
\newcommand{\ddzj}{\frac{\p}{\p z_j}}
\newcommand{\ddbzj}{\frac{\p}{\p \bz_j}}
\newcommand{\ddxj}{\frac{\p}{\p x_j}}
\newcommand{\ddyj}{\frac{\p}{\p y_j}}
\newcommand{\const}{\frac{1}{2\pi i}}
\newcommand{\sm}{\varepsilon}
\newcommand{\Resz}{\underset{z = 0}{\Res}}
\newcommand{\Resw}{\underset{w = 0}{\Res}}
\renewcommand{\div}{\mathrm{div}}
\newcommand{\Cl}{\mathrm{Cl}}
\newcommand{\Pic}{\mathrm{Pic}}
\newcommand{\holo}{holomorphic \,}
\newcommand{\mero}{meromorphic \,}
\newcommand{\Supp}{\mathrm{Supp}}



%%%%%%%%%%%%% new definitions for the positive mass paper %%%%%%%%%

\newcommand{\sperp}{{\scriptscriptstyle \perp}}

%%%%%%%%%%%%%%%%%%%%%%%

%%%%%%%%%%%%%%%%%%%%%%%%%%%%%%%%%%%%%%%%%%%%%



%
\begin{document}
%

\title{Notes on Riemann Surfaces}
\author{Ziquan Yang}


\date{\today}

\maketitle
 
\tableofcontents
%\setcounter{secnumdepth}{1} 

\setcounter{section}{0}
\section{Basic Geometric Properties}

\subsection{Some results from complex analysis}
\subsubsection{Local normal form}
Suppose $h : \Delta \to \Delta$ is a holomorphic function and $h(0) = 0$. 

\subsubsection{Fundamental theorem of algebra}
\subsection{Complex calculus on manifolds}

\subsubsection{Almost complex structures}
Suppose $U \subseteq \IC^n$ be an open subset. We want to decide when $f$ is holomorphic. We start by looking at the tangent bundles. Let $z_j = x_j + iy_j : U \to \IC$ and $x_j, y_j : U \to \IR$ are sections of $T^* U$. Here by $T^* U$ we mean the real tangent space of $U \subseteq \IC^n \iso \IR^{2n}$. Similarly we want to say $dz_j = dx_j + i dy_j$ is a section of $T^* U \tensor_\IR \IC = \Hom_\IR(TU, \IC)$. We also have $d\bz = dx_j - i dy_j$ as sections of $T^* U \tensor_\IR \IC$. We may often omit the subscript $\IR$, but it is important to keep in mind that when we are dealing with these situations $\tensor_\IR$ is assumed. Clearly $dx, dy$ and $dz, d\bz$ span the same space, since
$$ [dz, d\bz] = [dx, dy] \begin{bmatrix}
1 & 1 \\ i & -1 
\end{bmatrix}  $$
and matrix has determinant $-2i$. We may either use basis $dx_j, dy_j$'s or $dz_j, d\bz_j$'s. For the latter we set the dual basis
$$ \ddzj = \half( \ddxj - i \ddyj ), \, \ddbzj = \half (\ddxj + i \ddyj) $$
Now we may write 
$$ df = \sum_{j = 1}^n (\frac{\p f}{\p z_j} dz_j + \frac{\p f}{\p \bz_j} d
\bz_j) $$


We would like to endow the bundle $TU$ with the structure of a complex vector bundle, so we need to give an action of $\IC$ on fibers. We would like to say 
$$ i \ddxj = \ddyj $$
We define $J : TU \to TU$ by 
$$ J(\ddxj) = \ddyj, \, J(\ddyj) = -i \ddxj$$
Note that $J^2 = - 1$. Hence we obtain an action of $\IC \iso \IR[J]/(J^2 + 1)$ on $V_\IR$, where $V_\IR$ is a fiber of $TU$. 

\begin{lemma}
\label{decomp}
Let $V$ be a finite dimensional real vector space and $J : V \to V$ be a $\IR$-linear map satisfying $J^2 = -1$. Then $V \tensor \IC = V^1 \oplus V^2 \iso V \oplus i V $, where 
\begin{align*}
V' &= \{ v - i J v : v \in V \} \\
V'' &= \{ v + i J v : v \in V \} 
\end{align*}
Moreover $J|_{V'} = i, J|_{V''} = - i$. 
\end{lemma}
\begin{proof}
Straightforward verification. 
\end{proof}
\begin{remark} 
Let $J : V \to V$ be as above. We want to define a conjugate operation $V \tensor \IC \to V \tensor \IC$ given by $a + i b \mapsto a - i b$, where $a, b \in V$. Note that $\overline{V'} = V''$. 
\end{remark}

\paragraph{Example} Consider a fiber $T_p U$. Suppose we have already defined $J$. Then we can write $T_p U \tensor \IC = T_p' U \oplus T''_p U$, where $J$ acts on the first component by $i$ and the second by $- i$. 
$T_p' U$ is spanned by 
$$ \ddxj - i J(\ddyj) = \ddxj - i \ddyj = 2 \ddzj $$
and $T_p'' U $ is spanned by 
$$ \ddxj + i J(\ddyj) = \ddxj + i \ddyj = 2 \ddbzj $$

\begin{lemma}
The composite map $ \varphi' : V \to V \tensor \IC \iso V' \oplus V'' \to V' $ is complex-linear. Similarly, the composite map $\varphi'' : V \to V''$ is conjugate-linear. 
\end{lemma}
\begin{proof}
Recall that $i$ acts on $V$ as $J$, so being complex linear just means 
$$ \varphi'(J v) = \half(J v - i J^2 v) = \half(J v + i v) = \half i(v - i Jv) = i \varphi'(v) $$
\end{proof}

\paragraph{Example} Again we look at how this applies to a fiber $T_p U$. Consider $\varphi' : T_p U \to T'_p U$. We have 
\begin{align*}
\ddxj &\mapsto \ddzj \\
\ddyj &\mapsto \ddbzj 
\end{align*}
$J$ serves to permute the vectors in both columns. 

\subsubsection{Holomorphic functions}
\begin{definition}
A smooth function $f : U \to \IC$ is holomorphic at $p \in U$ if 
$$ Df : T_p U \to T_p \IC = \IC $$
is $\IC$-linear with respect to $J$. 
\end{definition}
Equivalently, we could also require that the induced map $T_p U \tensor_\IR \IC \to T_p \IC \tensor_\IR \IC$ commutes with $J$. 

Note that $J : V \to V$ induces a natural map $J : V^* \to V^*$. $J$ is self-ajoint in that 
$$ \< J \varphi , v \> = \< \varphi, J v \>, \forall \varphi \in V^*, v \in V $$
which can be verified by straigtforward computation. We let $J$ to act on the cotangent spaces as well, so by Lemma~\ref{decomp} we have a decomposition
$$ T^*_p U \tensor \IC = T^{1, 0}_p U \oplus T^{0, 1}_p U $$
where $J$ acts on $T^{1, 0}_p U$ as $i$ and on $T^{0, 1}_p U$ as $-i$. $T^{1, 0}_p U$ is spanned by $dz_1, \cdots, dz_n$ and $T^{0, 1}_p U$ is spanned by $d\bz_1, \cdots, d\bz_n$. 
Looking back to $f$, we see that $Df_p$ pulls $dw \in T_{f(p)}^* \IC \tensor_\IR \IC$ to $T^{1, 0}_p U$ and $d\bw$ to $T^{0, 1}_p U$. 

Write $w = f(z_1, \cdots, z_n)$. We see that 
$$
dw \mapsto df = \sum_{j = 1}^n \frac{\p f}{\p z_j} dz_j + \frac{\p f}{\p \bz_j} d\bz_j
$$  
and similarly $d \bw = d \overline{f}$. Recall that $d$ is a real operator, so $d \overline{f} = \overline{df}$. Now we make an important observation: 
\begin{align*}
(Df)^* \textit{ commutes with } J &\iff Df^*(dw) \textit{ has type }(1, 0)\\
&\iff df \textit{ has type }(1, 0) \\
&\iff \frac{\p f}{\p \bz_j} = 0, \, j = 1, \cdots, n
\end{align*}

\paragraph{Example} Let us look at the case $n = 1$. Write $f = u + iv$, where $u, v : U \to \IC$. Recall that 
$$ \frac{\p}{\p \bz} = \half( \frac{\p}{\p x} + i \frac{\p}{\p y}) $$ 
and hence 
$$ 2 \frac{\p f}{\p \bz} = (u_x - v_y) + i ( u_y + v_x ) $$
Therefore $f$ is holomorphic in our new definition if and only if it satisfies the Cauchy-Riemann equations. We can also readily make the following observation: 
\begin{lemma}
\label{regularcondition}
$f : U \to \IC$ is holomorphic if and only if it is holomophic in each coordinate, i.e. for all $(a_1, \cdots, a_n) \in U$ and $j = 1, \cdots, n$ the function 
$$ z \mapsto f(a_1, \cdots, a_{j - 1}, z, a_{j + 1}, \cdots, a_n) $$
is holomorphic. 
\end{lemma}

We say a function is \textit{analytic} if at each point it is given by a convergent power series. If $n = 1$, then it is well know that $f : U \to \IC$ is holomorphic if and only if it is analytic. We would like to extend this property to higher dimensions. Let $\Delta(z_j)$ denote copy of open disk of radius $r_j$ whose coordinate is labelled $z_j$. We have the following theorem. 

\begin{theorem}
A smooth function $f : \Delta(z_1) \times \cdots \times \Delta(z_n) \to \IC$ is holomorphic if and only if there is a power series 
$$ \sum_K c_K z_1^{k_1} \cdots z_n^{k_n}, \, K = (k_1, \cdots, k_n) \in \IN^n $$
that converges to $f(z_1, \cdots, z_n)$. 
\end{theorem}
\begin{proof}
Suppose $f$ is holomophic on $U$. Choose $(s_1, \cdots, s_n) \in \IR^n_{> 0}$ such that $|z_j| < s_j < r_j$. Then 
\begin{align*}
f(z_1, \cdots, z_n) &= \const  \int_{|w_n| = s_n} \frac{f(z_1, \cdots, z_{n-1}, w_n)}{w_n - z_n} dw_n \\
&= (\const)^2 \int_{|w_n| = s_n} \int_{|w_{n - 1}| = s_{n -1}} \frac{f(z_1, \cdots, w_{n-1}, w_n)}{(w_n - z_n)(w_{n - 1} - z_{n -1})} dw_{n-1} dw_n \\
&= \cdots \\
&= (\const)^n \int \int \cdots \int \frac{f(w_1, \cdots, w_n)}{\prod_{j = 1}^n ( w_j - z_j)} dw_1 \cdots dw_n
\end{align*}
Therefore we only need to expand the denominators: 
$$ \frac{1}{w_j - z_j} = \frac{1}{w_j} \sum_{n = 0}^\infty \frac{1}{w_j^n} z_j^n $$
The converse is straightforward by Lemma~\ref{regularcondition}. 
\end{proof}

\subsubsection{Differential forms}
Let $M$ be a $C^\infty$ manifold, $f : M \to \IC$ be a function. We write $f = u + i v$. $df = du + i dv$ is a $\IC$-valued $1$-form. It is a section of
$$ \Hom_\IR (TM, \IC) = \Hom_\IC ( TM \tensor \IC, \IC) \iso T^* M \tensor \IC \iso T^* M \oplus i T^*M $$
We can form the $k$th exterior product $\bigwedge^k T^* M$ and let $E^k(M)$ be the set of $C^\infty$ sections. Linear algebra says 
$$ \bigwedge^k (T^* U \tensor \IC) = \bigoplus_{p + q = k} (\bigwedge^p T^{1,0} U) \tensor (\bigwedge^q T^{0, 1} U) $$ 
Therefore $E_\IC^k(U) = \{ \textit{sections of } T^* U \tensor_\IR \IC \}$, 
$$ E^k_\IC (U) = \bigoplus_{p + q = k} E^{p, q}(U) $$ 

$E^{**}(U)$ is a bigraded algebra with operators $\p, \bp$. We define a map 
$$ d : E^0(X) \to E^{1, 0} \oplus E^{0, 1} $$ by $df = \p f + \bp f$, where $$\p f = \sum_{j = 1}^n \frac{\p f}{\p z_j} dz_j, \, \, \bp f = \sum_{j = 1}^n \frac{\p f}{\p \bz_j} d\bz_j$$ Note that $f : X \to \IC$ is a priori, only a $C^\infty$ function. It is holomorphic if and only if $\bp f = 0$. $\p, \bp$ extend to maps $E^{p,q}(X) \to E^{p+1, q}(X), E^{p, q}(X) \to E^{p, q+1}(X)$ and they still satisfy $\p^2 , \bp^2 = 0$. 

\begin{theorem}
Every complex manifold has a natural orientation. 
\end{theorem}

\begin{definition}
A holomorphic $1$-form on a Riemann surface $X$ is a closed $1$-form of type $(1, 0)$. 
\end{definition}

At the first spot of the definition, I find the requirement that the form must be closed a bit weird, but the following lemma tells us that closedness is exactly describing the condition of being ``locally holomorphic". 

\begin{lemma}
Let $f(z) dz$ be a $(1, 0)$-form on the disk $\Delta$. TFAE:
\begin{enumerate}[a.]
\item $f(z)dz$ is holomorphic. 
\item $f(z)$ is holomorphic. 
\item $\bp(f(z)dz) = 0$. 
\end{enumerate} 
\end{lemma}
\begin{proof}
\begin{align*}
d(fdz) &= df \wedge dz = (\frac{\p f}{\p z} dz + \frac{\p f}{\p \bz} d \bz) \wedge dz = \frac{\p f}{\p \bz} d \bz \wedge d z = 2i \frac{\p f}{\p \bz} dx \wedge dy
\end{align*}
Therefore 
$$ d(f(z)dz) = 0 \iff \frac{\p f}{\p \bz} = 0 \iff f \textit{ is holomorphic}$$
To show $a \iff c$, we first show that $\bp (dz) = 0$. Since $d^2 z = 0$, we know that $(\p + \bp) dz = 0$. $\p (dz)$ is a $(2, 0)$ form, but there is none. Therefore $\bp(dz) = 0$. Now 
$$ \bp (fdz) = \bp f \wedge dz = \frac{\p f}{\p \bz} d\bz \wedge dz = d(fdz) $$
\end{proof}
\begin{remark}
More generally, on a Riemann surface $X$, if $f : X \to \IC$ is a holomorphic function, then $df$ is a holomorphic $1$-form. 
\end{remark}

\paragraph{Example} Label the coordinates of $\IC^2$ by $(x, y)$. Let $X \subseteq \IC^2$ be the curve defined by $y^2 = p(x)$, where $p(x)$ is some square-free polynomial. Suppose the degree of $p(x)$ is $2g + 1$. We claim that 
$$ \frac{dx}{y}, \frac{x dx}{y}, \cdots, \frac{x^{g - 1} dx}{y} $$
Note that $x: X \to \IC$ is a holomorphic function, so $dx$ is a holomorphic $1$-form. Let $X' = X - y^{-1}(0)$. If 
$$ p(x) = \prod_{j = 0}^{2g} (x - a_j) $$
$y^{-1}(0) = \{ (a_j, 0), j = 0, 1, \cdots, 2g \}$. Clearly $x^n dx /y$ is holomorphic on $X'$ for all $n \ge 0$, so we are reaally trying to show that these forms are holomorphic at each $(a_j, 0)$. Without loss of generality, we assume $a_0 = 0$.  


\begin{definition}
The order of a meromorphic 1-form $\w$ at $p$ is $n$ if $\w = f(z) dz $ for some coordinate centered at $p$ and $\ord_p f(z) = n$. 
\end{definition}

\paragraph{Example} Let $X = \IC$. $\ord_0(dz/z) = -1$ and $\ord_0 (z^n dz) = n$. 

\subsubsection{Residues}

In complex analysis we can talk about residues of functions. More generally, suppose suppose $X$ is a Riemann surface and $f \in \fm(X)$. We may ask whether $\Res_p f$ well defined. This is in fact asking whether residues are invariant under change of coordiantes, but we readily see a counterexample. 

\paragraph{Example} Let $X = \IC$, $f = 1/z$. We may want to say $f$ has residue $1$ at the origin. However, under change of coordinate $w = az$ for some $a \neq 0$, the residue becomes $a$. 

Nonetheless, we can talk about residues of meromorphic forms. In the above example, if we instead look at $dz/z$, then 
$$ \frac{dw}{w} = \frac{a dw}{az} = \frac{dz}{z} $$
In complex analysis we know that residues can be used to compute line integrals. We may generalize this to a Riemann surface.
\begin{definition}
Suppose $\w$ is a holomorphic form on the punctured disk $\Delta^* = \{ z : 0 < |z| < 1 \}$. We can write $\w = f(z) dz$ where $f$ is a holomorphic function on $\Delta^*$ with expansion
$$ f = \sum_{ n = -\infty}^{\infty} c_n z^n $$ near $0$. Define 
$$ \Resz \w = c_{-1} = \const \int_{|z| = \sm} \w $$
for some $\sm < 1$. 
\end{definition}  
\begin{remark}
We readily make the observation that if $g$ is a holomorphic function on $\Delta^*$, then $$
\Resz dg = 0 $$
That is, an exact $1$-form always has residue zero. 
\end{remark}

\begin{proposition}
The residue does not depend on the choice of holomorphic coordinate. 
\end{proposition}
\begin{proof}
We start from a coordinate $z$ centered at the origin. Suppose $w$ is anther coordinate, so we can write $w = z \varphi(z)$ for some $\varphi \neq 0$ in a neighborhood. We hope to verify 
$$ \underset{z = 0}{\Res} (\sum_{n = - \infty}^\infty c_n w^n dw) = c_{-1} $$
We first show that 
$$ \Resz \frac{d \varphi}{\varphi} = 0 $$
Since $\varphi(0) \neq 0$, we can write $\varphi(z) = e^{\psi(z)}$ for some $\psi$ holomorphic near $0$. Now 
$$  \Resz \frac{d \varphi}{\varphi} = \Resz d \psi = 0 $$
Since $w = z \varphi(z)$, we have that 
$$ \frac{dw}{w} = \frac{dz}{z} + \frac{d\varphi}{\varphi} $$
The above implies 
$$
\Resz \frac{dw}{w} = \Resz \frac{dz}{z} + \Resz \frac{d\varphi}{\varphi} = 1 
$$
For $n \neq -1$, we see that $w^n dw$ is exact. Therefore they have residue zero. Now we are done.
\end{proof}

Now let $p \in X$ be a point and $f \in \fM(X)$ be a meromorphic function. Choose a coordinate $z$ centered at $p$. We can express $f = z^n \varphi(z)$ for some $n$ and $\varphi(0) \neq 0$. We observe that 
$$ \frac{df}{f} = \frac{n dz}{z} + \frac{d \varphi}{\varphi}$$
Hence we obtain the relationship:
\begin{lemma}
\label{resord}
$$ \Res_p \frac{df}{f} = \ord_p f $$
\end{lemma}

\begin{proposition}
\label{resandint}
For all holomorphic coordinate $z$ on $X$ centered at $p$ and all $\sm > 0$ small enough, 
$$\Res_p \w = \int_{|z| < \sm} \w $$
\end{proposition}

Now we prove the residue theorem: 
\begin{theorem}
Suppose $X$ is a compact Riemann surface. If $\w$ is a nonzero meromorphic $1$-form on $X$, then 
$$ \sum_{p \in X} \Res_p \w = 0 $$
\end{theorem}
\begin{proof}
The key idea is to use Proposition~\ref{resandint} and Stoke's theorem. 
Let $S$ be the set of poles of $\w$. It is a finite set since $X$ is compact. For each $p \in S$, choose a coordinate neighborhood $U_p$ with holomorphic coordinate $z_p$ centered at $p$. Without loss of generality, we may assume that $U_p$'s are disjoint and $U_p = \{z_p : |z_p| < 1 \}$. 

Set $X_\sm = X - \cup_p \{ z_p : |z_p| < \sm \} $ for some $\sm < 1$. It is a manifold with boundary. $\w$ is holomorphic on $X_\sm$, so $d \w|_{X_\sm} = 0$. By Stoke's theorem,
$$ 0 = \int d\w|_{X_\sm} = \int_{\p X_\sm} \w = \sum_{p \in X} - \int_{|z_p| = \sm} \w = - \sum_{p \in X} \Res_p \w $$ 
\end{proof}
\begin{remark}
The residue formula, together with our previous observation, provides another proof of Theorem~\ref{degmero}, since
$$ \sum_{p \in X} \ord_p f = \sum_{p \in X} \Res_p \frac{df}{f} = 0 $$
\end{remark}


\section{Theory of Divisors}

\subsection{Basic properties}
\subsubsection{Definitions}
Let $X$ be a Riemann surface. A function $D : X \to \IZ$ can also be regarded as a formal sum $\sum_p D(p)[p]$. 
\begin{definition}
A divisor on $X$ is a locally finite formal sum $D = \sum_p n_p [p]$.
\end{definition}
By locally finite we mean that the set $\{ p \in X : D(p) \neq 0\}$ is discrete, so when $X$ is compact, a divisor has to be a finite sum. 
The set of divisors, which we denote by $\Div(X)$, is a group with the obvious additive structure. It is a free abelian group generated by the points in $X$. In other words, 
$$ \Div(X) = \bigoplus_{p \in X} \IZ $$


We want to define $L(D) = \{ f \in \fM(X): (f) + D \ge 0 \} \cup \{ 0 \}$. Notation: $\Div(0) = \sum_p \infty [p]$ and hence $\Div(0) \ge D$ for all $D$. 

We use divisors to keep track of zeroes and poles of functions and poles. If $f \in \fM(X)$, we define 
$$ \div(f) = \sum_{p \in X} \ord_p (f) [p] $$
which we may also denote simply by $(f)$. 

\paragraph{Example} $X = \IC$, $f(z) = e^z - 1$. Then $f$ has a zero of order $1$ at each $2n \pi i$, so 
$$ \div(f) = \sum_{n \in \IZ} [2n \pi i] $$

Similarly if $\w$ is a meromorphic 1-form on $X$, 
we define $$ \div(\w) = \sum_{p \in X} \ord_p(\w)[p]$$

\paragraph{Example} Let $X = \IP^1$, $\w = dz/z$. On $\IC$, $\w$ has a simple pole at $0$. At $\infty$ we need to use coordinate $w = 1/z$ and 
$$ \frac{dz}{z} = \frac{-w^{-2} dw}{w^{-1}} = - \frac{dw}{w} $$
$dw/w$ has a simple pole at $w = 0$, so does $dz/z$ at $\infty$. Therefore on $\IP^1$, 
$$ \div(\frac{dz}{z}) = -[0] - [\infty] $$

For a meromorphic 1-form $\w$ we may also define the residue divisor: 
$$ \Res(\w) = \sum_{p \in X} \Res_p(\w) [p] $$
By Lemma~\ref{resord} we see that $\Res(df/f) = \div(f)$. 

\begin{proposition}
If $f, g \in \fM(X)$ and $\w \neq 0$ is a meromorphic 1-form, then 
\begin{enumerate}
\item $\div(fg) =\div(f) + \div(g)$. 
\item $\div(1/g) = - \div(g)$. 
\item $\div(f\w) = \div(f) + \div(\w)$. 
\end{enumerate}
\end{proposition}
\begin{proof}
Straightforward. 
\end{proof}

We can split $\div(f)$ as $\div_0(f) - \div_\infty (f)$, where $\div_0(f)$ is the zero divisor and $\div_\infty(f)$ is the pole divisor. 

\subsubsection{Degree of divisors}
We assume $X$ is a compact Riemann surface. Define a function $\deg : \Div(X) \to \IZ$ by 
$$ \deg(\sum_{p \in X} n_p [p]) = \sum_{p \in X} n_p $$

\paragraph{Example} $\deg(\div(f)) = 0$ and $\deg(Res(\w)) = 0$ for all $f \in \fM(X)$ and $\w$ a meromorphic 1-form. 

\begin{definition}
A principle divisor on a Riemann surface $X$ is a divisor which is the divisor of some $f \in \fM(X)$. 
\end{definition}
\begin{definition}
The divisor class group of $X$ is 
$$ \Cl(X) = \Div(X)/ \PDiv(X)$$ 
\end{definition}
Clearly $\deg$ descends to a map $\Cl(X) \to \IZ$. 

\paragraph{Example} Let $X = \IP^1$, $\w = dz/z$. 
$ \div(dz/z) = -[0] - [ \infty ]$,
so $\deg(\div(dz/z)) = -2$. 
\paragraph{Example} Let $X = \IC/\Lambda$, $\w = dz$. $dz$ has no zeroes and no poles. Therefore $\deg(\div(\w)) = 0$. 

\begin{proposition}
Any two canonical divisors on any Riemann surface differ by a principle divisor.  
\end{proposition}
\begin{proof}
Suppose $\w_1, \w_2$ are nonzero meromorphic 1-forms. We claim that $\w_2 = h \w_1$ for some $h \in \fM(X)$. Locally we may write $\w_1 = f_1(z) dz$ and $\w_2 = f_2(z) dz$, so $h = f_2/f_1$. We readily check that the definition of $h$ is independent of coordinates. Hence $\div(\w_2) = \div(\w_1) + \div(h)$. 
\end{proof}


\begin{proposition}
$$\deg : \Cl(\IP^1) \to \IZ$$ is an isomorphism. 
\end{proposition}
\begin{proof}
The map is clearly surjective. Now any degree zero divisor can be written as a linear combination of $([0] - [p])$'s, so we only need to show that these lie in $\PDiv(X)$. When $p = a \in \IC$, we have 
$$ \div(\frac{z}{z - a}) = [0] - [a] $$
When $p = \infty$, we have 
$$ \div(z) = [0] - [\infty] $$  
\end{proof}

\begin{theorem}
If $X$ is a compact Riemann surface of genus $g$, then the degree of the canonical divisor is $2g - 2$. 
\end{theorem}
\begin{proof}
We need to assume that we have a non-constant meromorphic function, which we will prove later using analysis. Recall that a meromorphic function gives us a map $f : X \to \IP^1$ and 
$$ \frac{df}{f} = f^*(\frac{dz}{z}) $$
\end{proof}



\subsection{Holomorphic line bundles}

\begin{definition}
Let $X$ be a Riemann surface. A holomorphic line bundle $L$ on $X$ is a topological space $L$, together with a projection $\pi : L \to X$ with the following properties: 
\begin{enumerate}
\item $\pi^{-1}(p) = L_p$ is a complex vector space of dimension $1$ for all $p \in X$. 
\item Local triviality: There is an open cover $\{ U_\alpha\}$ of $X$ such that there is a homeomorphism $\varphi_\alpha : \IC \times U_\alpha \to L|_U$ that is $\IC$-linear on each fiber and 
\[
\begin{tikzcd}
\IC \times U_\alpha \arrow{dr}{p_2} \arrow{rr}{\varphi_\alpha} && L|_{U_\alpha} \arrow{dl}{\pi} \\
{} & U 
\end{tikzcd}
\]
\item On each intersection $U_{\alpha \beta} = U_\alpha \cap U_\beta$, the function $g_{\alpha \beta} : U_{\alpha \beta} \to \IC^*$ such that 
$s_\beta(z) = s_\alpha(z) g_{\alpha \beta}(z)$ is holomorphic. 
\end{enumerate}
\end{definition}

If $U_{\alpha \beta \gamma} = U_\alpha \cap U_\beta \cap U_\gamma \neq \emptyset$, $g_{\alpha \gamma} = g_{\alpha \beta} g_{\beta \gamma}$. We say these transitions functions satisfy the cocycle condition. Since line bundles are locally trivial, it is exactly encoded by its transition functions. That is, if we start from an open cover and transition functions that satisfy the cocycle condition, then we can glue up a line bundle.  

\paragraph{Example} We call $T^{1, 0} X$ the holomorphic cotangent bundle. On coordinate neighborhood $(U, z)$, the section $dz$ trivializes the bundle. If $\{ (U_\alpha, z_\alpha) \}$ is a covering of $X$, then the transition functions are given by 
$$ g_{\alpha \beta} = \frac{d z_\beta}{d z_\alpha}, \text{ since } dz_\beta = \frac{d z_\beta}{d z_\alpha} d z_\alpha $$
Similarly, $T'X$, the holomorphic tangent bundle, is locally trivialized by $\p/\p z_\alpha$. It is the dual bundle of $T^{1, 0} X$, and its transition functions are 
$$ g'_{\alpha \beta} = \frac{d z_\alpha}{d z_\beta} $$

\begin{definition}
A holomorphic section of a holomorphic line bundle $\pi : L \to X$ is a section $s : X \to L$ such that for each $U \subseteq X$, $\sigma : U \to L|_U \iso U \times \IC$ a local trivialization, $s = f \sigma$, $f$ is holomorphic. 
\end{definition}

\begin{definition}
A meromorphic section of $\pi: L \to X$ is a holomorphic section of $s : X' \to L|_{X'}$ for some $X' = X - A$, where $A$ is a discrete and closed subset of $X$. 
\end{definition}

We call holomorphic (resp. meromorphic) sections of $T^{1, 0} X$ holomorphic (resp. meromorphic) 1-forms. 

A line bundle is trivial if and only if it has a nowhere vanishing holomorphic section. 

\paragraph{Example} If $X = \IC/\Lambda$, then $T^{1, 0}X$ is trivial since clearly $dz$ is a nowhere vanishing holomorphic section. If $X = \{ y^2 = p(x) \} \subseteq \IC^2$, where $p(x)$ is some square free cubic polynomial, then $dx/y$ is a trivializing section. 

\begin{notation}
$$H^0(X, L) = \{ \text{holomorphic sections of } L \}$$
\end{notation} 

\begin{definition}
The divisor of a meromorphic section of $L$ is 
$$ \div(s) = \sum_{ p \in X} \ord_p (s) [p] $$ 
where $\ord_p(s)$ is defined as $\ord_p(f)$ for some $s = f \sigma$, $\sigma$ is local trivializing section. 
\end{definition} 

Any clearly any two sections of a line bundle differ by a \mero function. Therefore if we knew that every \holo line bundle had a section, then every line bundle $L$ would give an element of $\Cl(X) = \Div(X)/\PDiv(X)$. 

\begin{definition}
We define the Picard group 
$$\Pic(X) = \{ \text{\holo line bundles over }X\}/\text{isomorphism} $$
where group law is given by $[L_1] + [L_2] = [L_1 \tensor_\IC L_2]$. 
\end{definition}

The transition functions of a tensor product is the product of those of individual line bundles. 

Now we wish to prove 
\begin{theorem}
If $X$ is compact, then $\Pic(X) \iso \Cl(X)$. 
\end{theorem}

But we start from some propositions. 
\begin{proposition}
If $D \in \Div(X)$, then there is a \holo line bundle and $L_D \to X$ and a \mero section $s_D$ with $D = \div(s_D)$. 
\end{proposition}
\begin{proof}
Suppose $D = \sum_p n_p [p]$. Let $ A = \Supp(D) = \{ p : n_p \neq 0 \}$. Let $(U_p, z_p)$ be a coordinate disk centered at $p \in A$. Let $U_0 = X - A$ and $\sU = \{ U_0 \} \cup \{ U_p : p \in A \}$. Without loss of generality, we may assume that these $U_p$'s are disjoint, so that there are no triple intersections in $\sU$. Let $f_p = z_p^{n_p}$ on $U_p$ and $f_0 = 1$ on $U_0$. On the open cover $\fU$ we give transition functions $g_{p0} = z_p^{n_p}$. They automatically satisfy the cocycle condition and hence we have constructed a line bundle $L_D \to X$. 
\end{proof}

\begin{proposition}
If $L \to X$ is a line bundle with transition functions $g_{\alpha \beta} : U_{\alpha \beta} \to \IC^*$, then every section $\sigma$ of $L$ is of the form $\sigma|_{U_\alpha} : f_\alpha s_\alpha$ where $f_\alpha = g_{\alpha \beta} f_\beta$. 
\end{proposition}
\begin{proof}
On $U_{\alpha \beta}$, we have that 
$$ s = f_\alpha s_\alpha = f_\beta s_\beta = f_\beta s_\alpha g_{\alpha \beta} $$
Therefore $f_\alpha = g_{\alpha \beta} f_\beta$. 
\end{proof}


\begin{proposition}
Suppose $D \in \Div(X)$. Then $f \cdot s(D) \in H^0(X, \sL_D)$ if and only if $f \in L(D)$.
\end{proposition}



\begin{proposition}
If $X$ is compact and $\deg D = 0$, then $L(D) \neq 0$ if and only if $D$ is principal. 
\end{proposition}









\end{document}